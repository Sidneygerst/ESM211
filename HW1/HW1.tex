\documentclass[]{article}
\usepackage{lmodern}
\usepackage{amssymb,amsmath}
\usepackage{ifxetex,ifluatex}
\usepackage{fixltx2e} % provides \textsubscript
\ifnum 0\ifxetex 1\fi\ifluatex 1\fi=0 % if pdftex
  \usepackage[T1]{fontenc}
  \usepackage[utf8]{inputenc}
\else % if luatex or xelatex
  \ifxetex
    \usepackage{mathspec}
  \else
    \usepackage{fontspec}
  \fi
  \defaultfontfeatures{Ligatures=TeX,Scale=MatchLowercase}
\fi
% use upquote if available, for straight quotes in verbatim environments
\IfFileExists{upquote.sty}{\usepackage{upquote}}{}
% use microtype if available
\IfFileExists{microtype.sty}{%
\usepackage{microtype}
\UseMicrotypeSet[protrusion]{basicmath} % disable protrusion for tt fonts
}{}
\usepackage[margin=1in]{geometry}
\usepackage{hyperref}
\hypersetup{unicode=true,
            pdftitle={Homework 1},
            pdfauthor={Sidney Gerst},
            pdfborder={0 0 0},
            breaklinks=true}
\urlstyle{same}  % don't use monospace font for urls
\usepackage{color}
\usepackage{fancyvrb}
\newcommand{\VerbBar}{|}
\newcommand{\VERB}{\Verb[commandchars=\\\{\}]}
\DefineVerbatimEnvironment{Highlighting}{Verbatim}{commandchars=\\\{\}}
% Add ',fontsize=\small' for more characters per line
\usepackage{framed}
\definecolor{shadecolor}{RGB}{248,248,248}
\newenvironment{Shaded}{\begin{snugshade}}{\end{snugshade}}
\newcommand{\KeywordTok}[1]{\textcolor[rgb]{0.13,0.29,0.53}{\textbf{#1}}}
\newcommand{\DataTypeTok}[1]{\textcolor[rgb]{0.13,0.29,0.53}{#1}}
\newcommand{\DecValTok}[1]{\textcolor[rgb]{0.00,0.00,0.81}{#1}}
\newcommand{\BaseNTok}[1]{\textcolor[rgb]{0.00,0.00,0.81}{#1}}
\newcommand{\FloatTok}[1]{\textcolor[rgb]{0.00,0.00,0.81}{#1}}
\newcommand{\ConstantTok}[1]{\textcolor[rgb]{0.00,0.00,0.00}{#1}}
\newcommand{\CharTok}[1]{\textcolor[rgb]{0.31,0.60,0.02}{#1}}
\newcommand{\SpecialCharTok}[1]{\textcolor[rgb]{0.00,0.00,0.00}{#1}}
\newcommand{\StringTok}[1]{\textcolor[rgb]{0.31,0.60,0.02}{#1}}
\newcommand{\VerbatimStringTok}[1]{\textcolor[rgb]{0.31,0.60,0.02}{#1}}
\newcommand{\SpecialStringTok}[1]{\textcolor[rgb]{0.31,0.60,0.02}{#1}}
\newcommand{\ImportTok}[1]{#1}
\newcommand{\CommentTok}[1]{\textcolor[rgb]{0.56,0.35,0.01}{\textit{#1}}}
\newcommand{\DocumentationTok}[1]{\textcolor[rgb]{0.56,0.35,0.01}{\textbf{\textit{#1}}}}
\newcommand{\AnnotationTok}[1]{\textcolor[rgb]{0.56,0.35,0.01}{\textbf{\textit{#1}}}}
\newcommand{\CommentVarTok}[1]{\textcolor[rgb]{0.56,0.35,0.01}{\textbf{\textit{#1}}}}
\newcommand{\OtherTok}[1]{\textcolor[rgb]{0.56,0.35,0.01}{#1}}
\newcommand{\FunctionTok}[1]{\textcolor[rgb]{0.00,0.00,0.00}{#1}}
\newcommand{\VariableTok}[1]{\textcolor[rgb]{0.00,0.00,0.00}{#1}}
\newcommand{\ControlFlowTok}[1]{\textcolor[rgb]{0.13,0.29,0.53}{\textbf{#1}}}
\newcommand{\OperatorTok}[1]{\textcolor[rgb]{0.81,0.36,0.00}{\textbf{#1}}}
\newcommand{\BuiltInTok}[1]{#1}
\newcommand{\ExtensionTok}[1]{#1}
\newcommand{\PreprocessorTok}[1]{\textcolor[rgb]{0.56,0.35,0.01}{\textit{#1}}}
\newcommand{\AttributeTok}[1]{\textcolor[rgb]{0.77,0.63,0.00}{#1}}
\newcommand{\RegionMarkerTok}[1]{#1}
\newcommand{\InformationTok}[1]{\textcolor[rgb]{0.56,0.35,0.01}{\textbf{\textit{#1}}}}
\newcommand{\WarningTok}[1]{\textcolor[rgb]{0.56,0.35,0.01}{\textbf{\textit{#1}}}}
\newcommand{\AlertTok}[1]{\textcolor[rgb]{0.94,0.16,0.16}{#1}}
\newcommand{\ErrorTok}[1]{\textcolor[rgb]{0.64,0.00,0.00}{\textbf{#1}}}
\newcommand{\NormalTok}[1]{#1}
\usepackage{graphicx,grffile}
\makeatletter
\def\maxwidth{\ifdim\Gin@nat@width>\linewidth\linewidth\else\Gin@nat@width\fi}
\def\maxheight{\ifdim\Gin@nat@height>\textheight\textheight\else\Gin@nat@height\fi}
\makeatother
% Scale images if necessary, so that they will not overflow the page
% margins by default, and it is still possible to overwrite the defaults
% using explicit options in \includegraphics[width, height, ...]{}
\setkeys{Gin}{width=\maxwidth,height=\maxheight,keepaspectratio}
\IfFileExists{parskip.sty}{%
\usepackage{parskip}
}{% else
\setlength{\parindent}{0pt}
\setlength{\parskip}{6pt plus 2pt minus 1pt}
}
\setlength{\emergencystretch}{3em}  % prevent overfull lines
\providecommand{\tightlist}{%
  \setlength{\itemsep}{0pt}\setlength{\parskip}{0pt}}
\setcounter{secnumdepth}{0}
% Redefines (sub)paragraphs to behave more like sections
\ifx\paragraph\undefined\else
\let\oldparagraph\paragraph
\renewcommand{\paragraph}[1]{\oldparagraph{#1}\mbox{}}
\fi
\ifx\subparagraph\undefined\else
\let\oldsubparagraph\subparagraph
\renewcommand{\subparagraph}[1]{\oldsubparagraph{#1}\mbox{}}
\fi

%%% Use protect on footnotes to avoid problems with footnotes in titles
\let\rmarkdownfootnote\footnote%
\def\footnote{\protect\rmarkdownfootnote}

%%% Change title format to be more compact
\usepackage{titling}

% Create subtitle command for use in maketitle
\newcommand{\subtitle}[1]{
  \posttitle{
    \begin{center}\large#1\end{center}
    }
}

\setlength{\droptitle}{-2em}

  \title{Homework 1}
    \pretitle{\vspace{\droptitle}\centering\huge}
  \posttitle{\par}
    \author{Sidney Gerst}
    \preauthor{\centering\large\emph}
  \postauthor{\par}
      \predate{\centering\large\emph}
  \postdate{\par}
    \date{January 17, 2020}


\begin{document}
\maketitle

\begin{enumerate}
\def\labelenumi{\arabic{enumi}.}
\tightlist
\item
  Three reasons why plants or animals might be patchily distributed:

  \begin{itemize}
  \tightlist
  \item
    Plants don't have the ability to move away. They may drop seeds that
    fall directly by where they live forcing themselves to be clumped or
    patchy. Because they stay in one place, patches will not vary year
    to year.\\
  \item
    Animals tend to live with their own species for protection such a
    herd of american bison. Herds of animals, while they will stay
    together, will not stay in one are, so patches will vary year to
    year.
  \item
    Suitable habitat is essential for an ecosystem. Nutrient
    concentration may be a factor for plants where patched soil
    concentrations effect where the plants grow. Animals will follow
    that pattern as grazers will feast on the plant patches and
    carnivores will stay close to their prey. Patches may vary year to
    year.
  \end{itemize}
\end{enumerate}

\textbf{Eureka Dune Grass}

\begin{enumerate}
\def\labelenumi{\arabic{enumi}.}
\setcounter{enumi}{1}
\item
\end{enumerate}

\begin{Shaded}
\begin{Highlighting}[]
\CommentTok{#Read in Swallenia Data}
\NormalTok{swallenia <-}\StringTok{ }\KeywordTok{read_csv}\NormalTok{(}\StringTok{"Swallenia.csv"}\NormalTok{)}

\CommentTok{#Perform two sample t-test}
\NormalTok{swallenia_ttest <-}\StringTok{ }\KeywordTok{t.test}\NormalTok{(swallenia}\OperatorTok{$}\NormalTok{count_}\DecValTok{2009}\NormalTok{, swallenia}\OperatorTok{$}\NormalTok{count_}\DecValTok{2010}\NormalTok{, }\DataTypeTok{var.equal =} \OtherTok{TRUE}\NormalTok{)}

\NormalTok{swallenia_ttest}
\end{Highlighting}
\end{Shaded}

\begin{verbatim}
## 
##  Two Sample t-test
## 
## data:  swallenia$count_2009 and swallenia$count_2010
## t = -0.81791, df = 20, p-value = 0.423
## alternative hypothesis: true difference in means is not equal to 0
## 95 percent confidence interval:
##  -140.07803   61.16893
## sample estimates:
## mean of x mean of y 
##  61.90909 101.36364
\end{verbatim}

Based on an \(\alpha\) = 0.05 the difference in mean abundance of
swallenia does not have a significant change between 2009 and 2010 (p =
0.42). This alpha is appropriate because it is a 5\% risk of concluding
a difference exists when there is no actual difference.

\begin{enumerate}
\def\labelenumi{\arabic{enumi}.}
\setcounter{enumi}{2}
\item
\end{enumerate}

\begin{Shaded}
\begin{Highlighting}[]
\CommentTok{#Paired t-test}
\NormalTok{swallenia_pairedtest <-}\StringTok{ }\KeywordTok{t.test}\NormalTok{(swallenia}\OperatorTok{$}\NormalTok{count_}\DecValTok{2009}\NormalTok{, swallenia}\OperatorTok{$}\NormalTok{count_}\DecValTok{2010}\NormalTok{, }\DataTypeTok{paired =} \OtherTok{TRUE}\NormalTok{)}

\NormalTok{swallenia_pairedtest }
\end{Highlighting}
\end{Shaded}

\begin{verbatim}
## 
##  Paired t-test
## 
## data:  swallenia$count_2009 and swallenia$count_2010
## t = -2.4508, df = 10, p-value = 0.03421
## alternative hypothesis: true difference in means is not equal to 0
## 95 percent confidence interval:
##  -75.324830  -3.584261
## sample estimates:
## mean of the differences 
##               -39.45455
\end{verbatim}

The true difference in means of abundance of swallenia between 2009 and
2010 does not equal to 0, so there is a difference in abundance (p =
0.03).

\begin{enumerate}
\def\labelenumi{\arabic{enumi}.}
\setcounter{enumi}{3}
\item
\end{enumerate}

The second analysis with the paired t-test is more accurate than a two
sample t-test because each plot is not independent of eachother between
the two years.

\begin{enumerate}
\def\labelenumi{\arabic{enumi}.}
\setcounter{enumi}{4}
\item
\end{enumerate}

Conservation and continuous monitoring efforts should be focused on
swallenia dune grass species. Between the years 2009 and 2010 there was
a change in dune grass population. Through a Paired t-test calculation,
a p-value of 0.03 was found. This means that there is a 3\% chance of
not finding a difference in the means, low enough to assume there is a
difference. Continued monitoring and conservation of dune grass should
take place. More years with data would be effective to show how trends
in growth will change.

\textbf{Yellowstone Grizzly Bears}

\begin{enumerate}
\def\labelenumi{\arabic{enumi}.}
\setcounter{enumi}{5}
\item
\end{enumerate}

\begin{Shaded}
\begin{Highlighting}[]
\NormalTok{grizzly <-}\StringTok{ }\KeywordTok{read_csv}\NormalTok{(}\StringTok{"grizzlydata.csv"}\NormalTok{) }

\CommentTok{#Filter years 1959 to 1968}
\NormalTok{grizzly5968 <-}\StringTok{ }\NormalTok{grizzly }\OperatorTok
\StringTok{  }\KeywordTok{filter}\NormalTok{(Year }\OperatorTok{>=}\StringTok{ "1959"} \OperatorTok{&}\StringTok{ }\NormalTok{Year }\OperatorTok{<=}\StringTok{ "1968"}\NormalTok{)}


\CommentTok{#perform linear regression}
\NormalTok{grizzly5968_model <-}\StringTok{ }\KeywordTok{lm}\NormalTok{(N }\OperatorTok{~}\StringTok{ }\NormalTok{Year, }\DataTypeTok{data =}\NormalTok{ grizzly5968)}


\NormalTok{grizzly5968_model}
\end{Highlighting}
\end{Shaded}

\begin{verbatim}
## 
## Call:
## lm(formula = N ~ Year, data = grizzly5968)
## 
## Coefficients:
## (Intercept)         Year  
##   1543.0000      -0.7636
\end{verbatim}

\begin{Shaded}
\begin{Highlighting}[]
\CommentTok{# N(grizzlies) = 1543 - 0.7636(Year)}

\CommentTok{#summary function}
\NormalTok{griz59_sum <-}\StringTok{ }\KeywordTok{summary}\NormalTok{(grizzly5968_model)}
\NormalTok{griz59_sum}
\end{Highlighting}
\end{Shaded}

\begin{verbatim}
## 
## Call:
## lm(formula = N ~ Year, data = grizzly5968)
## 
## Residuals:
##     Min      1Q  Median      3Q     Max 
## -3.0364 -1.5591 -0.1273  1.5182  3.5455 
## 
## Coefficients:
##              Estimate Std. Error t value Pr(>|t|)  
## (Intercept) 1543.0000   472.9413   3.263   0.0115 *
## Year          -0.7636     0.2409  -3.170   0.0132 *
## ---
## Signif. codes:  0 '***' 0.001 '**' 0.01 '*' 0.05 '.' 0.1 ' ' 1
## 
## Residual standard error: 2.188 on 8 degrees of freedom
## Multiple R-squared:  0.5568, Adjusted R-squared:  0.5014 
## F-statistic: 10.05 on 1 and 8 DF,  p-value: 0.01319
\end{verbatim}

\begin{Shaded}
\begin{Highlighting}[]
\CommentTok{#standard error: 2.188,  p-value: 0.01319}

\CommentTok{#Visualize the data}
\NormalTok{girzzly5968_plot <-}\StringTok{ }\KeywordTok{ggplot}\NormalTok{(grizzly5968, }\KeywordTok{aes}\NormalTok{(}\DataTypeTok{x =}\NormalTok{ Year, }\DataTypeTok{y =}\NormalTok{ N)) }\OperatorTok{+}
\StringTok{  }\KeywordTok{geom_point}\NormalTok{() }\OperatorTok{+}
\StringTok{  }\KeywordTok{geom_smooth}\NormalTok{(}\DataTypeTok{method =}\NormalTok{ lm, }\DataTypeTok{se =}\NormalTok{ T, }\DataTypeTok{size =} \FloatTok{0.5}\NormalTok{)}\OperatorTok{+}
\StringTok{  }\KeywordTok{theme_bw}\NormalTok{() }\OperatorTok{+}
\StringTok{  }\KeywordTok{scale_x_continuous}\NormalTok{(}\DataTypeTok{expand =} \KeywordTok{c}\NormalTok{(}\DecValTok{0}\NormalTok{,}\DecValTok{0}\NormalTok{), }\DataTypeTok{limits =} \KeywordTok{c}\NormalTok{(}\DecValTok{1958}\NormalTok{, }\DecValTok{1969}\NormalTok{)) }\OperatorTok{+}
\StringTok{  }\KeywordTok{labs}\NormalTok{(}\DataTypeTok{x =} \StringTok{"Year"}\NormalTok{, }\DataTypeTok{y =} \StringTok{"Number of Grizzlies"}\NormalTok{)}


\NormalTok{girzzly5968_plot}
\end{Highlighting}
\end{Shaded}

\includegraphics{HW1_files/figure-latex/unnamed-chunk-3-1.pdf}

Based on a linear regression, the population of grizzly bears decreases
0.7636 bears per year between the years 1959 and 1968. This model has
significant evidence that there is predictability (S = 2.188, p =
0.013).

\begin{enumerate}
\def\labelenumi{\arabic{enumi}.}
\setcounter{enumi}{6}
\item
\end{enumerate}

\begin{Shaded}
\begin{Highlighting}[]
\CommentTok{#filter years 1969 to 1978}
\NormalTok{grizzly6978 <-}\StringTok{ }\NormalTok{grizzly }\OperatorTok
\StringTok{  }\KeywordTok{filter}\NormalTok{(Year }\OperatorTok{>=}\StringTok{ "1969"} \OperatorTok{&}\StringTok{ }\NormalTok{Year }\OperatorTok{<=}\StringTok{ "1978"}\NormalTok{ )}

\CommentTok{#perform linear regression }
\NormalTok{grizzly6978_model <-}\StringTok{ }\KeywordTok{lm}\NormalTok{(N }\OperatorTok{~}\StringTok{ }\NormalTok{Year, }\DataTypeTok{data =}\NormalTok{ grizzly6978)}

\NormalTok{grizzly6978_model}
\end{Highlighting}
\end{Shaded}

\begin{verbatim}
## 
## Call:
## lm(formula = N ~ Year, data = grizzly6978)
## 
## Coefficients:
## (Intercept)         Year  
##   1532.3758      -0.7576
\end{verbatim}

\begin{Shaded}
\begin{Highlighting}[]
\CommentTok{#N(grizzlies) = 1532.3758 - 0.7576(Year)}

\CommentTok{#summary statistics}
\NormalTok{griz69_sum <-}\StringTok{ }\KeywordTok{summary}\NormalTok{(grizzly6978_model)}
\NormalTok{griz69_sum}
\end{Highlighting}
\end{Shaded}

\begin{verbatim}
## 
## Call:
## lm(formula = N ~ Year, data = grizzly6978)
## 
## Residuals:
##     Min      1Q  Median      3Q     Max 
## -4.6788 -0.9439  0.2303  1.4955  3.5939 
## 
## Coefficients:
##              Estimate Std. Error t value Pr(>|t|)  
## (Intercept) 1532.3758   536.3641   2.857   0.0213 *
## Year          -0.7576     0.2718  -2.787   0.0237 *
## ---
## Signif. codes:  0 '***' 0.001 '**' 0.01 '*' 0.05 '.' 0.1 ' ' 1
## 
## Residual standard error: 2.469 on 8 degrees of freedom
## Multiple R-squared:  0.4927, Adjusted R-squared:  0.4293 
## F-statistic:  7.77 on 1 and 8 DF,  p-value: 0.02365
\end{verbatim}

\begin{Shaded}
\begin{Highlighting}[]
\CommentTok{#S = 2.469, p = 0.023}

\CommentTok{#visuallize data}
\NormalTok{grizzly6978_plot <-}\StringTok{ }\KeywordTok{ggplot}\NormalTok{(grizzly6978, }\KeywordTok{aes}\NormalTok{(}\DataTypeTok{x =}\NormalTok{ Year, }\DataTypeTok{y =}\NormalTok{ N)) }\OperatorTok{+}
\StringTok{  }\KeywordTok{geom_point}\NormalTok{() }\OperatorTok{+}
\StringTok{  }\KeywordTok{geom_smooth}\NormalTok{(}\DataTypeTok{method =}\NormalTok{ lm, }\DataTypeTok{se =}\NormalTok{ T, }\DataTypeTok{size =} \FloatTok{0.5}\NormalTok{)}\OperatorTok{+}
\StringTok{  }\KeywordTok{theme_bw}\NormalTok{() }\OperatorTok{+}
\StringTok{  }\KeywordTok{scale_x_continuous}\NormalTok{(}\DataTypeTok{expand =} \KeywordTok{c}\NormalTok{(}\DecValTok{0}\NormalTok{,}\DecValTok{0}\NormalTok{), }\DataTypeTok{limits =} \KeywordTok{c}\NormalTok{(}\DecValTok{1968}\NormalTok{, }\DecValTok{1979}\NormalTok{)) }\OperatorTok{+}
\StringTok{  }\KeywordTok{labs}\NormalTok{(}\DataTypeTok{x =} \StringTok{"Year"}\NormalTok{, }\DataTypeTok{y =} \StringTok{"Number of Grizzlies"}\NormalTok{)}


\NormalTok{grizzly6978_plot}
\end{Highlighting}
\end{Shaded}

\includegraphics{HW1_files/figure-latex/unnamed-chunk-4-1.pdf}

\begin{Shaded}
\begin{Highlighting}[]
\CommentTok{#comparing the two models: ttest on the slopes}
\NormalTok{grizzly_ttest <-}\StringTok{ }\KeywordTok{t.test}\NormalTok{(grizzly5968}\OperatorTok{$}\NormalTok{N, grizzly6978}\OperatorTok{$}\NormalTok{N, }\DataTypeTok{var.equal =}\NormalTok{ T)}
\NormalTok{grizzly_ttest}
\end{Highlighting}
\end{Shaded}

\begin{verbatim}
## 
##  Two Sample t-test
## 
## data:  grizzly5968$N and grizzly6978$N
## t = 4.4242, df = 18, p-value = 0.0003275
## alternative hypothesis: true difference in means is not equal to 0
## 95 percent confidence interval:
##  3.308286 9.291714
## sample estimates:
## mean of x mean of y 
##      43.6      37.3
\end{verbatim}

Based on a linear regression, the population of grizzly bears decreases
0.7576 bears per year between the years 1969 and 1978. This model has
significant evidence that there is predictability (S = 2.469, p =
0.023). This is a significantly slower decline than the previous ten
years (p = 0.0003).

\begin{enumerate}
\def\labelenumi{\arabic{enumi}.}
\setcounter{enumi}{7}
\item
\end{enumerate}

\begin{Shaded}
\begin{Highlighting}[]
\CommentTok{#filter the rest of the data}
\NormalTok{grizzly79 <-}\StringTok{ }\NormalTok{grizzly }\OperatorTok
\StringTok{  }\KeywordTok{filter}\NormalTok{(Year }\OperatorTok{>=}\StringTok{ "1979"}\NormalTok{)}

\CommentTok{#perform linear regression}
\NormalTok{grizzly79_model <-}\StringTok{ }\KeywordTok{lm}\NormalTok{(N }\OperatorTok{~}\StringTok{ }\NormalTok{Year, }\DataTypeTok{data =}\NormalTok{ grizzly79)}

\NormalTok{grizzly79_model}
\end{Highlighting}
\end{Shaded}

\begin{verbatim}
## 
## Call:
## lm(formula = N ~ Year, data = grizzly79)
## 
## Coefficients:
## (Intercept)         Year  
##   -6211.288        3.154
\end{verbatim}

\begin{Shaded}
\begin{Highlighting}[]
\CommentTok{# N(grizzly) = -6211.288 + 3.14(Years)}

\CommentTok{#summary statistics}
\NormalTok{griz79_sum <-}\StringTok{ }\KeywordTok{summary}\NormalTok{(grizzly79_model)}
\NormalTok{griz79_sum}
\end{Highlighting}
\end{Shaded}

\begin{verbatim}
## 
## Call:
## lm(formula = N ~ Year, data = grizzly79)
## 
## Residuals:
##      Min       1Q   Median       3Q      Max 
## -18.4035  -4.0140   0.2947   3.8316  14.1333 
## 
## Coefficients:
##               Estimate Std. Error t value Pr(>|t|)    
## (Intercept) -6211.2877   650.9227  -9.542 3.06e-08 ***
## Year            3.1544     0.3274   9.634 2.67e-08 ***
## ---
## Signif. codes:  0 '***' 0.001 '**' 0.01 '*' 0.05 '.' 0.1 ' ' 1
## 
## Residual standard error: 7.817 on 17 degrees of freedom
## Multiple R-squared:  0.8452, Adjusted R-squared:  0.8361 
## F-statistic: 92.81 on 1 and 17 DF,  p-value: 2.667e-08
\end{verbatim}

\begin{Shaded}
\begin{Highlighting}[]
\CommentTok{#Visualize }
\NormalTok{grizzly79_plot <-}\StringTok{ }\KeywordTok{ggplot}\NormalTok{(grizzly79, }\KeywordTok{aes}\NormalTok{(}\DataTypeTok{x =}\NormalTok{ Year, }\DataTypeTok{y =}\NormalTok{ N)) }\OperatorTok{+}
\StringTok{  }\KeywordTok{geom_point}\NormalTok{() }\OperatorTok{+}
\StringTok{  }\KeywordTok{geom_smooth}\NormalTok{(}\DataTypeTok{method =}\NormalTok{ lm, }\DataTypeTok{se =}\NormalTok{ T, }\DataTypeTok{size =} \FloatTok{0.5}\NormalTok{)}\OperatorTok{+}
\StringTok{  }\KeywordTok{theme_bw}\NormalTok{() }\OperatorTok{+}
\StringTok{  }\KeywordTok{scale_x_continuous}\NormalTok{(}\DataTypeTok{expand =} \KeywordTok{c}\NormalTok{(}\DecValTok{0}\NormalTok{,}\DecValTok{0}\NormalTok{), }\DataTypeTok{limits =} \KeywordTok{c}\NormalTok{(}\DecValTok{1978}\NormalTok{, }\DecValTok{1998}\NormalTok{)) }\OperatorTok{+}
\StringTok{  }\KeywordTok{labs}\NormalTok{(}\DataTypeTok{x =} \StringTok{"Year"}\NormalTok{, }\DataTypeTok{y =} \StringTok{"Number of Grizzlies"}\NormalTok{)}


\NormalTok{grizzly79_plot}
\end{Highlighting}
\end{Shaded}

\includegraphics{HW1_files/figure-latex/unnamed-chunk-5-1.pdf}

The trend in the population continued to change, but in the other
direction from the previous two decades. The population size of
grizzlies from 1979 to 1997 increases 3.14 bears per year. The trend in
population growth is positive. This model has significant evidence that
there is predictability (S = 7.817, p \textless{} 0.001).

\begin{enumerate}
\def\labelenumi{\arabic{enumi}.}
\setcounter{enumi}{8}
\item
\end{enumerate}

During the 1960's there was a decline in the population of female
grizzly bears. Starting in 1969, the decline started to slow down, where
the population of grizzlies did not decline as quickly as they did the
decade before. The dump was taken out of the park in 1968. Starting in
the year 1979, the trend in grizzly bear population switches from
decreasing to increasing. The only the females with cubs are counted in
this analysis, so it is safe to say that keeping the dump within the
park closed has a positive effect on the female bear population.


\end{document}
